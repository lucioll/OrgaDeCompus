%%%%%%%%%%%%%%%%%%%%%%%%%%%%%%%%%%%%%%%%%%%%%%%%%%%%%%%%%%%%%%%%%%%%%%%%%%%%%%%
% Definici�n del tipo de documento.                                           %
% Posibles tipos de papel: a4paper, letterpaper, legalpapper                  %
% Posibles tama�os de letra: 10pt, 11pt, 12pt                                 %
% Posibles clases de documentos: article, report, book, slides                %
%%%%%%%%%%%%%%%%%%%%%%%%%%%%%%%%%%%%%%%%%%%%%%%%%%%%%%%%%%%%%%%%%%%%%%%%%%%%%%%
\documentclass[a4paper,10pt]{article}
\usepackage{listings}
\usepackage[left=3cm,right=3cm,top=2cm,bottom=2cm]{geometry}
\usepackage{amsmath,amssymb}
%%%%%%%%%%%%%%%%%%%%%%%%%%%%%%%%%%%%%%%%%%%%%%%%%%%%%%%%%%%%%%%%%%%%%%%%%%%%%%%
% Los paquetes permiten ampliar las capacidades de LaTeX.                     %
%%%%%%%%%%%%%%%%%%%%%%%%%%%%%%%%%%%%%%%%%%%%%%%%%%%%%%%%%%%%%%%%%%%%%%%%%%%%%%%

% Paquete para inclusi�n de gr�ficos.
\usepackage{graphicx}
\usepackage{float}

% Paquete para definir la codificaci�n del conjunto de caracteres usado
% (latin1 es ISO 8859-1).
\usepackage[latin1]{inputenc}

% Paquete para definir el idioma usado.
\usepackage[spanish]{babel}



% T�tulo principal del documento.
\title{		\textbf{TP2 - Memorias cach�}}

% Informaci�n sobre los autores.
\author{	Lucio Lopez Lecube, \textit{Padr�n Nro. 96583}                     \\
            \texttt{luciolopezlecube@gmail.com }                                              \\
            Santiago Alvarez Juli�, \textit{Padr�n Nro. 99522}                     \\
            \texttt{ santiago.alvarezjulia@gmail.com }                                              \\
            Bautista Canavese, \textit{Padr�n Nro. 99714}                     \\
            \texttt{ bauti-canavese@hotmail.com }                                              
 \\[2.5ex]
            \normalsize{Grupo Nro.   - 1er. Cuatrimestre de 2018}                       \\
            \normalsize{66.20 Organizaci�n de Computadoras}                             \\
            \normalsize{Facultad de Ingenier�a, Universidad de Buenos Aires}            \\ }


\begin{document}

% Inserta el t�tulo.
\maketitle

% Quita el n�mero en la primer p�gina.
\thispagestyle{empty}

\newpage
\tableofcontents
\newpage

\section{Introducci�n}

Simulaci�n de una memoria cach� asociativa por conjuntos de dos v�as, de 1KB de capacidad, bloques de 32 bytes, pol�tica de reemplazo LRU
y pol�tica de escritura WT/notWA. Se asume que el espacio de direcciones es de 12 bits, y hay entonces una memoria principal a simular con un tama�o
de 4KB. 

\section{Problem�tica a desarrollar}
El programa a escribir, en lenguaje C, recibir� un archivo de texto plano, en donde se encuentran las instrucciones a realizar de la siguiente manera:
\begin{verbatim}
	W 0, 16
	R 0
	R 1024
	R 3074
	W 8, 12
	R 8
	R 8
	W 3072, 255
	R 0
	MR
\end{verbatim}

\begin{itemize}
\item Los comandos de la forma \texttt{R dddd} lee el dato dddd y lo imprime.
\item Los comandos de la forma \texttt{W dddd, vvv} escriben en la direcci�n dddd el valor vvv.
\item Los comandos de la forma \texttt{MR} imprime el miss rate actual del programa ejecutado sobre la cache.
\end{itemize}


Adem�s, implementando las siguientes primitivas:

\begin{verbatim}
void init()
unsigned char read_byte(int address)
int write_byte(int address, unsigned char value)
unsigned int get_miss_rate()
\end{verbatim}

\section{Detalles de implementaci�n}
	
	\subsection{Estructura Cache}
	Division de address:	
		
	\begin{itemize}
	\item Offset de palabra de 3 bits, ya en cada bloque se encuentran 8 lineas.
	\item Offset de bloque de 2 bits, al ser de 4 bytes la palabra.
	\item Tag formado por 3 bits, contiene los bits resantes de la direcci�n.
	\item Index formado por otros 4 bits para seleccionar entre los 16 bloques.
	\end{itemize}
	
	Por lo tanto, la direcci�n es interpretada con un offset formado por los primeros 5 bits menos significativos y los 7 bits subsiguientes conforman el index y el tag.
	
	\subsection{Salida est�ndar}
Se muestra por salida est�ndar seg�n el comando solicitado por el usuario. En la siguiente secci�n $Ejemplos$ se podr�n apreciar los flags implementados del programa y como ejecutarlos.

\section{Ejemplos}
Con la opci�n -h se vera los distintos falgs disponibles para ejecutar el programa.
\begin{verbatim}
$ 
Usage:
  cache -h
  cache -V
  cache [options] filename
Options:
  -h, --help    Prints usage information.
  -V, --version Prints version information.
  -i, --input   Path to input file.

Example:
./cache  input_file
./cache -i input_file
\end{verbatim}



\section{Compilaci�n}
Realizar \texttt{make} sobre la carpeta en donde se encuentran los archivos, generara un ejecutable \texttt{cache} . 

Github: \texttt{https://github.com/lucioll/OrgaDeCompus.git} dentro de la carpeta \texttt{TP2}.

%\section{Conclusi�n}

	%Mediante el estudio y simulaci�n de una cache en C, pudimos observar y comprender su funcionamiento dentro de la ejecuci�n de un programa, como se relaciona con el CPU y la memoria principal.

	%Asimismo entendimos la importancia de elecci�n en cuanto al tamanio de memoria asociada al la cache (bloques y vias), adem�s de la pol�tica de reemplazo utilizada en caso de haber conflictos.

	%Finalmente concluimos que no necesariamente colocando una cache a un programa este se va a ejecutar mas eficientemente, influye fuertemente como este implementada, cuantas v�as utiliza, y otros factores mencionados anteriormente.


% Citas bibliogr�ficas.
\begin{thebibliography}{99}

\bibitem{HEN00} Hennessy, John L. and Patterson, David A., Computer Architecture: A
Quantitative Approach, Third Edition, 2002.


\end{thebibliography}

\end{document}
